\documentclass[a4paper,12pt]{article}
\usepackage{geometry}
\geometry{a4paper, margin=1in}

\usepackage{amsmath, amssymb}

\usepackage{tikz}

\usepackage{graphicx}
\usepackage{caption}

\usepackage{fancyhdr}
\usepackage{graphicx}
\usepackage{titlesec}
\usepackage{titling}
\usepackage{xcolor}
\usepackage{hyperref}
\hypersetup{
    colorlinks=true,
    linkcolor=blue,
    filecolor=magenta,      
    urlcolor=cyan,
}

% Header and Footer
\pagestyle{fancy}
\fancyhf{}
\fancyhead[L]{\includegraphics[height=0.8cm]{logo.jpg}} % Include your institution logo
\fancyhead[C]{\footnotesize \textbf{\courseName}}
\fancyhead[R]{\includegraphics[height=0.8cm]{ionq_logo_simple.png}}

\fancyfoot[L]{\projectName}
\fancyfoot[C]{Page \thepage}
\fancyfoot[R]{\today}

% Title formatting
\pretitle{\begin{center}\LARGE \bfseries}
\posttitle{\end{center}\vspace{0.5cm}}
\preauthor{\begin{center}\normalsize}
\postauthor{\end{center}}
\predate{\begin{center}\small}
\postdate{\end{center}\vspace{1cm}}

% Section formatting
\titleformat{\section}[block]{\Large\bfseries}{\thesection}{1em}{}
\titleformat{\subsection}[block]{\large\bfseries}{\thesubsection}{1em}{}
\titleformat{\subsubsection}[block]{\bfseries}{\thesubsubsection}{1em}{}

% Custom commands for student details
%\newcommand{\studentName}{Hyunseong Kim}
%\newcommand{\studentID}{20195048}
\newcommand{\courseName}{2024 IonQ summer Mentoring}
%\newcommand{\courseCode}{MM4020}
\newcommand{\assignmentTitle}{OptTrot}
%\newcommand{\dueDate}{June 30, 2024}

\newcommand{\projectName}{OpTrot}

\newtheorem{theorem}{Theorem}
\newtheorem{lemma}{Lemma}
\newtheorem{definition}{Definition}
\newtheorem{observation}{Observation}


\begin{document}

% Title Section
\begin{center}
%    %\includegraphics[width=0.15\textwidth]{logo.png}\par\vspace{1cm} % Include your institution logo
%    {\scshape \courseName \par}
%    {Final Assignment \par}
%    \vspace{0.5cm}

    \vspace{0.5cm}
    {\Large\bfseries \assignmentTitle \par}
    {\large Optimized Trotter Circuit Library \par}
    \vspace{1cm}
    {
    \noindent
    \begin{minipage}{0.45\textwidth}
        \centering
        \textbf{Memebers}

        Hyunseong Kim

        Hanseo Kim
        
        Gaya Yun
    \end{minipage}
    \begin{minipage}{0.45\textwidth}
        \centering
        \textbf{Mentor}

        Sayomeo Ray
    \end{minipage}
    }
%    {\itshape \studentName \par}
%    {Student ID: \studentID \par}
%    \vspace{0.5cm}
%    {\large \today \par}
    \begin{abstract}
    Abstract
    \end{abstract}
\end{center}


%\tableofcontents
%wpage

\section{Introduction}

Trotterization is a standard method used to implement a time evolution operator 
by combining several local Hamiltonian evolution operators.
By using the method, we can expect the approximated operator closed to the original
operator, even the local terms did not commute with each other.

\begin{equation}
    \lim_{n \rightarrow \infty} (e^{A/2} e^{B/2})^n = e^{A+B}
\end{equation}

In quantum computing, the method has an advantage of preserving the local structure of 
the Hamiltonian on a dynamic circuit \cite{childs_theory_2021}.
However, Trotterization method increases circuit depth with linear order by number of Pauli terms.
If the time evolution was an ultimate goal to achieve in quantum circuit, 
it could be meaningful, but in the most algorithms and applications, time evolution 
is just a part of the whole process. 
Thus, reducing techniques are significant to apply the quantum computer to general taskes.
In addition, increased circuit depth for reducing Trotter error yields 
inefficient costs in NISQ era, which makes the algorithm into less practical one.

By the limitation, there are many alternative methods to implement a time evolution operator 
with shorter depth circit than Trotterization, such as 
linear combination of unitary(LCU) method\cite{dewolf2023quantumcomputinglecturenotes}, Qubitization\cite{Low_2019}, 
Taylorization\cite{PhysRevLett.114.090502}, and Fractional query\cite{Berry_2014}.
Such methods make the evolution circuit more practical, however, they loose 
identity of the given system, especially the cases, when the given hamiltonian is nearly commute
or local observable was a dominant feature\cite{childs_theory_2021}. 

Using Trotter error schema with commuting terms, 
we can analyze the overall error by applying order of local terms.


\subsection{Trotter Error by applying order}

It is well known that the exponential mapping error is represented with Baker Campbell Hausdorff formula.
Usually, the formula is not written with commutator form, Childs et al proved that the error term 
as a function of sequential commutator of local terms\cite{childs_theory_2021}.

\begin{equation}
    O(\alpha t^2)
\end{equation}

The results of Childs et al allow us to calculate 
the error boundary more precisely including a physical structure 
of the given Hamiltonian.

For example, let a given Hamiltonian was $H = c_i P_i + c_j P_j$.

\begin{equation}
    \exp(-it (c_i) P_i) \exp(-it (c_j) P_j) = \exp(- it (c_i P_i + c_j P_j)) + O (\alpha_{com}t^2)
\end{equation}

then the leading coefficient becomes $\alpha_{com} = \begin{cases}
    c_i + c_j & \mbox{ if } [P_i, P_j] = 0 \\
    c_i - c_j & \mbox{ if } [P_i, P_j] \neq 0 \\
\end{cases}$.

It is affected by coefficients, their size, and sign, and commutation property.
In the above example, we cannot observe the commutation and anti-commutation
effect, since, if they were commuting to each other, the $O(\alpha_{com} t^2) = 0$.
Let us expand the system to more general case.
Suppose that the given Hamiltonian has two representations,

\begin{align}
    H = H_1 + H_2 + H_3  \\
    H =  c_1 P_1 + c_2 P_2 + c_3 P_3 + c_4 P_4 + c_5 P_5\\
    H_1 = c_1 P_1 + c_3 P_3 \\
    H_2 = c_2 P_2\\
    H_3 = c_4 P_4 + c_5 P_5
\end{align}

where, $[H_i, H_j] \neq 0,$ and 
$[P_k, P_l] \neq 0$ if $P_k \in H_i, P_l \in H_j, i \neq j$.

\begin{align}
    \Pi_{l=1}^5 \exp(- i t (c_l P_l)) = \exp(-it H) + O(\alpha_{com 1} t^2)\label{eq:pauli_evolve}\\
    \Pi_{k=1}^3 \exp(- i t (H_k)) = \exp(-it H) + O(\alpha_{com 2} t^2) \label{eq:evolve_commute}
\end{align}


Following the $q=1$ order expansion, then in the first order, the bound error 
coefficients are reduced to 

\begin{align}
    \alpha_{com1} = 2(|| c_1 c_2 [P_1, P_2]|| + || c_1 c_4 [P_1, P_4]|| + || c_1 c_5 [P_1, P_5]||& \\
    + || c_2 c_3 [P_2, P_3]|| + || c_2 c_4 [P_2, P_4]|| + || c_2 c_5 [P_2, P_5]||& \\
    + || c_3 c_4 [P_3, P_4]|| + || c_3 c_5 [P_3, P_5]||)&\\
    \alpha_{com2} = 2(|| [H_1, H_2]|| + || [H_1, H_3]|| + || [H_2, H_3]||)& \\
\end{align}


\begin{align}
    0.5 \alpha_{com1} &= ||c_1 c_2|| + ||c_3 c_2|| + ||c_1 c_4|| + ||c_2 c_4|| + ||c_1 c_5|| + ||c_2 c_5||  + ||c_2 c_4|| + ||c_2 c_5||\\
    0.5 \alpha_{com2} &= ||c_1 c_2 + c_3 c_2|| + ||c_1 c_4 + c_2 c_4 + c_1 c_5 + c_2 c_5||  + ||c_2 c_4 + c_2 c_5||
\end{align}


Therefore, by the distribution of $\{c_i\}$ and commutation relationship of the local terms,
the constructed error rate vary.
However, Eq(\ref{eq:evolve_commute}) is just a re-ordering of Eq(\ref{eq:pauli_evolve}), since 

\begin{align}
    \exp(- i t (H_1)) &= \exp(- i t (c_1 P_1)) \exp(- i t (c_3 P_3)) \\ 
    \exp(- i t (H_2)) &= \exp(- i t (c_2 P_2))\\
    \exp(- i t (H_3)) &= \exp(- i t (c_4 P_4)) \exp(- i t (c_5 P_5)) \\ 
\end{align}

This is not et al, local Hamiltonian consist of mutually commuting Pauli terms provides us a lot of freedom
to optimize the circuit. 
Whatever we switch and reorder the Pauli terms in the applying trotter circuit,
if the larger structure, $\exp(-it H_j) = \Pi_k \exp(-it c_jk P_jk)$, was preserved then, 
the trotter error would be bounded while we reduce the number of gates in the implementation.
Thus, we don't have to choose one between trotter error and large gate error
to optimize the evolution circuit.

\textbf{Note}: The partitioning the hamiltonian into large local terms does not yields the optimized 
order in every situation. 
The term $\alpha$ only indicates upper bound of the error. So, the minimum trotter error could be 
worse than the arbitrary applying case. 
In here, the main focus is reducing errors associated with gate operations by using minimum
number of gates.

\section{Mutually Commuting Paritition}


\section{Optimizing a circuit with commuting pairs}

Clique: optimal condition: sum $ \Sigma_{i} c_i \approx 0 $.

\subsection{Pauli Frame method}

The current stage does not treat the routine finding 
optimal path search in Pauli Frame space.
It is because that we have not much tools to deal 
find the path and weight between the given frames,
excepting the dynamics programming approach.

$I, Z$ family would be a good example to see the complxity
of the problem.
Consider that the local term of the Hamiltonian composite of $Z$ family.
Then, for a given frame $B$ and suppose that we have to make next frame to have $P = IIZZ$.
Since, we don't have to consider terms containing $X, Y$, we can simply write the term 
as single column, and the required entanglement gate is $CX$ gate only.

\begin{equation}
    B = \begin{bmatrix}
        ZZZI\\
        ZZIZ\\
        ZIZZ\\
        IZZZ
    \end{bmatrix}
    \rightarrow B' = \begin{bmatrix}
        IIZZ\\
        \dots\\
        \dots\\
        \dots
    \end{bmatrix}
\end{equation}


The answer is shotly applying a $CX$ gate on line 1 and 2.
Now, let the representation as binary vector
\begin{equation}
B = \begin{bmatrix}
    ZZZI\\
    ZZIZ\\
    ZIZZ\\
    IZZZ
\end{bmatrix} = \begin{bmatrix}
    1110\\
    1101\\
    1011\\
    0111
\end{bmatrix} = 
\begin{bmatrix}
    \vec{w}_1\\
    \vec{w}_2\\
    \vec{w}_3\\
    \vec{w}_4
\end{bmatrix}
\end{equation}

Then the problem becomes the next statement.

\begin{quotation}
    \textit{Find the minimum size subgroup $\{w_k\} \subset B$ whose XOR products is $P$
    where, }

    \begin{equation}
        P = \vec{w}_1^\wedge \vec{w}_2 
    \end{equation}

\end{quotation}



More simply, it is equivalent with finding a binary vector $\vec{x} \in \{0, 1\}^N$ of 

\begin{equation}
    P = \otimes_{i=1}^N x_i \& \vec{w}_i
\end{equation}

There is not proper polynomial algorithm of the problem.(\textit{We might verify this statement!})
In addition, the general local Hamiltonian terms are harder than the only Z-family cases.
Every attempt would be an approximation unless the new frameworks are developed.
\section{Conclusion}

In the report, we overlook the trotter error affected by order and hamiltonian structure 
in $n$-th order Suzukit-Trotter formula.

%Bibliography
\bibliographystyle{unsrt}  
\bibliography{references}  

\end{document}
